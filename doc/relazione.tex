\documentclass[a4paper, 10pt]{article}
\usepackage[legalpaper, margin=1in]{geometry}
\usepackage{graphicx}
\usepackage{bookmark}
\usepackage{hyperref}
\usepackage{amssymb}
\usepackage{upgreek}
\usepackage{float}
\usepackage[italian]{babel}

\begin{document}
    \begin{titlepage}
        \begin{center}
            \includegraphics[width=0.3\textwidth]{./assets/LogoUNIPD.png}\\

            \vspace*{1cm}

            \Huge
            \textbf{Documento di specifiche}

            \vspace{1cm}

            \LARGE
            \textbf{Klaudio Merja}

            \vspace{0.2cm}

            \large
            \textbf{Mat. 2075538}

            \vspace{0.2cm}

            \url{https://github.com/klamerja/SensorFlowUNIPD}

            \vspace{1.5cm}

            \tableofcontents
            \vfill

            \LARGE
            \textbf{SensorFlow}

            \normalsize
            \vspace{0.8cm}
            Progetto in itinere di Programmazione ad Oggetti\\
            LT in Informatica\\
            Università degli Studi di Padova
        \end{center}
    \end{titlepage}
    \section{Introduzione}
    \textbf{SensorFlow} è un software di gestione per sensori in ambito domotico. Ogni sensore è identificato tramite un UUID ed è caratterizzato da un nome, dalla tipologia e dalla distribuzione dei dati generati.\\
    Le tipologie di sensori per cui l'applicazione fornisce supporto sono:
    \begin{itemize}
        \item \textbf{Temperatura e umidità}: permette di analizzare la temperatura (in °C) e l'umidità (in percentuale)
        \item \textbf{Pressione atmosferica} (in hPa - ettopascal) 
        \item \textbf{Elettricità}: permette di analizzare il consumo istantaneo (in W - watt) e la tensione elettrica (in V - volt)
        \item \textbf{Qualità dell'aria}: permette di analizzare i livelli di CO2 (in ppm - parti per milione), il PM2.5 ed il PM10 (in $\upmu$g/m$^3$)
    \end{itemize}
    Le operazioni principali che l'applicazione permette di svolgere sono:
    \begin{itemize}
        \item aggiunta/rimozione dei sensori
        \item modifica delle informazioni relative ai singoli sensori
        \item visualizzazione dei dati generati
    \end{itemize}
    Una delle caratteristiche fondamentali del software è quella di poter visualizzare i dati generati dal sensore in tempo reale.\\
    I dati, per fornire una simulazione del sensore, sono generati secondo una tipologia di distribuzione tra le seguenti:
    \begin{itemize}
        \item Casuale
        \item Uniforme
        \item Gaussiana
    \end{itemize}
    L'utente ha la possibilità di decidere quale distribuzione adottare per ogni singolo sensore e di modificarla in un secondo momento.
    \section{Descrizione del modello}
    Il modello logico è strutturato in due parti: la prima parte comprende le classi che descrivono i vari sensori utilizzabili all'interno dell'applicativo, mentre la seconda parte si occupa di creazione, lettura e aggiornamento del file JSON che si occupa del salvataggio dei sensori.
    \begin{figure}[H]
        \centering
        \includegraphics[scale=0.3]{assets/UML.png}
        \caption{Diagramma delle classi dei sensori}
    \end{figure}
    \texttt{AbstractSensor} è la classe base astratta che rappresenta le informazioni comuni a tutti i sensori che possono essere creati, ovvero il nome, il timer, l'identificatore univoco UUID e la tipologia di distribuzione. Oltre ai relativi metodi \emph{getter} e \emph{setter} per le varie variabili d'istanza, 
    è presente il metodo \texttt{onTimerTimeout}, che si occupa di effettuare delle azioni ad ogni \emph{timeout} emesso dal timer: solitamente, le azioni che vengono performate sono quelle di aggiornamento dei valori dei dati che vengono generati dal sensore, oltre a cancellare valori dalle eventuali serie che risultano non necessarie ai fini del funzionamento del software e, in particolare, per la generazione del grafico. Le classi figlie di \texttt{AbstractSensor} sono:
    \begin{itemize}
        \item \texttt{PressureSensor}
        \item \texttt{TempHumidity}
        \item \texttt{AirQualitySensor}
        \item \texttt{ElectricitySensor}
    \end{itemize}

    \section{Polimorfismo}
    \section{Persistenza dei dati}
    \section{Funzionalità implementate}
    \section{Rendicontazione ore}
    \begin{center}
        \begin{tabular}{|l|c|c|}
            \hline
            \textbf{Attività} & \textbf{Ore Previste} & \textbf{Ore effettive} \\
            \hline
            Studio e progettazione & 10 & 10 \\
            \hline
            Sviluppo del codice del modello & 10 & 10 \\
            \hline
            Studio del framework Qt & 10 & 10 \\
            \hline
            Sviluppo del codice della GUI & 10 & 10 \\
            \hline
            Test e debug & 10 & 10 \\
            \hline
            Stesura della relazione & 10 & 10 \\
            \hline
            \textbf{Totale} & 10 & \textbf{10} \\
            \hline
        \end{tabular}
    \end{center}
\end{document}